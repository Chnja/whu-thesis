% \documentclass[type = bachelor]{whu-thesis}
\documentclass[type = master,class = academic]{whu-thesis}
% \documentclass[type = doctor]{whu-thesis}
% type: 学位类型,可选项为 bachelor, master, doctor
% class: 学位类别,可选项为 academic, professional
% showframe: 显示页面布局框架

% 以下仅列举了部分可能用到的设置选项,更多用法请参考文档《whu-thesis.pdf》

% \PassOptionsToPackage{gbnamefmt = lowercase}{biblatex} % 英文作者姓名不强制大写

\whusetup{
  info = {
    title      = {论文题目,楷体一号}, % 标题,可使用 \\ 手动换行
    title*     = {A QoS-aware Self-Adaptive System Approach in Service Computing Environment},
    department  = {数学与统计学院},
    department* = {School of Mathematics and Statistics},
    author     = {张三},
    author*    = {Xxxxxx Xxxxx},
    student-id = {2021202012345},
    supervisor  = {李某某},
    supervisor* = {Xxx Xxxxx},
    academic-title  = {教授},
    academic-title* = {Prof},
    % supervisor-outer = {王某某}, % 校外导师(非必填)
    % academic-title-outer = {高级工程师}, % 校外导师职称(非必填)
    subject = {英语}, % 学科名称(非必填)
    major   = {英国语言文学},
    major*  = {English language and literature},
    research-area  = {英语语言学},
    research-area* = {Operator Theory},
    year = 2024,
    month = 4,
    % clc = , % 分类号
    % udc = ,
    keywords  = {\LaTeX{}, 毕业论文, 模版, 武汉大学}, % 中文关键词
    keywords* = {\LaTeX{}, Thesis, Template, Wuhan~University}, % 英文关键词
  },
  style = {
    % 字体相关选项
    font = termes, % 西文字体,可选项为 default, times, xits, termes
    math-font = termes, % 数学字体,可选项为 default, xits, termes
    cjk-font = mac, % 中文字体,可选项为 windows, mac, fandol(Linux/Overleaf/TexPage), sourcehan, none
    % cjk-fakefont = true, % 使用伪粗体与伪斜体
    % 参考文献及引用相关选项
    bib-backend = bibtex, % 参考文献引擎,可选项为 bibtex, biblatex
    % bib-style = numerical, % 参考文献样式,可选项为 numerical, author-year
    % cite-style = <>, % 引用样式(自定义)
    bib-resource = {ref/bachelor-refs.bib}, % 参考文献数据源
    % 页面相关选项
    % chapter-page-header = true, % 章节首页是否有页眉
    % bachelor-encover = true, % 本科毕业论文英文封面
    % library, % 图书馆模式(去掉论文中所有的空白页)
    % license, % 使用授权协议书
    % fullwidth-stop = true, % 句号样式
    % footnote-style = <>, % 脚注编号样式
    % abstract-keywords-type  = blankline, % 摘要与关键词之间样式,可选项为 blankline, newline, vfill
    % abstract-keywords-type* = blankline, % 摘要与关键词之间样式,可选项为 blankline, newline, vfill
  }
}
\whumodule{algorithm2e}
\begin{document}
% \raggedbottom % 使得空白都置于前一页底部,可参考https://github.com/whutug/whu-thesis/issues/276

\tableofcontents % 目录
% \listoffigures % 图目录
% \listoftables % 表目录

% 符号表
% \begin{notation}
%   $\omega_n$ & $n$-维欧氏空间中单位球的表面积 \\
%   $\alpha_n$ & $n$-维欧氏空间中单位球的体积 \\
% \end{notation}

\mainmatter

\chapter{测试}

\section{文字测试}

这是武汉大学学位论文模版,欢迎使用。

This is Wuhan University thesis template, welcome to use!

\section{公式}

\subsection{算符、希腊字母}

\[\sum\prod\int\iint\alpha\beta\gamma\xi\zeta\eta\epsilon\varepsilon\theta\vartheta
  \phi\varphi\psi\]


\subsection{几类数学字母表}

\begin{itemize}
  \item \verb|\mathcal|: $\mathcal{ABCDEFGHIJKLMNOPQRSTUVWXYZ}$
  \item \verb|\mathscr|: $\mathscr{ABCDEFGHIJKLMNOPQRSTUVWXYZ}$
  \item \verb|\mathbb|: $\mathbb{ABCDEFGHIJKLMNOPQRSTUVWXYZ}$
\end{itemize}


\subsection{(不)带编号单行公式}

Use \texttt{equation} environment:
\begin{equation}
  a^2 + b^2 = c^2.
\end{equation}

Use \texttt{equation*} environment or \texttt{\textbackslash[...\textbackslash]}:
\[ a^2 + b^2 = c^2.\]

\subsection{(不)带编号多行公式}

Use \texttt{align} environment:
\begin{align}
  S_n & = 1 + 2 + \cdots + n \\
      & = \frac12 n(n+1).
\end{align}

Use \texttt{align*} environment:
\begin{align*}
  T_n & = 1^3 + 2^3 + \cdots + n^3 \\
      & = \biggl(\frac{n(n+1)}{2}\biggr)^2 \\
      & = S_n^2.
\end{align*}

\subsection{矩阵}

\[\begin{pmatrix}
  a_{11} & a_{22} & a_{33} \\
  a_{21} & a_{22} & a_{23} \\
  a_{31} & a_{32} & a_{33} \\
\end{pmatrix} \quad
\begin{vmatrix}
  a_{11} & a_{22} & a_{33} \\
  a_{21} & a_{22} & a_{23} \\
  a_{31} & a_{32} & a_{33} \\
\end{vmatrix} \quad
\begin{bmatrix}
  a_{11} & a_{22} & a_{33} \\
  a_{21} & a_{22} & a_{23} \\
  a_{31} & a_{32} & a_{33} \\
\end{bmatrix} \quad
\begin{Bmatrix}
  a_{11} & a_{22} & a_{33} \\
  a_{21} & a_{22} & a_{23} \\
  a_{31} & a_{32} & a_{33} \\
\end{Bmatrix}\]

\section{脚注测试}

测试 \footnote{眼看他起朱楼,眼看他宴宾客,眼看他楼塌了。这青苔碧瓦堆,俺曾睡风流觉,将五十年兴亡看饱。
金粉未消亡,闻得六朝香,满天涯烟草断人肠。怕催花信紧,风风雨雨,误了春光。}

测试 \footnote[3]{君不见,左纳言,右纳史,朝承恩,暮赐死。行路难,不在水,不在山,只在人情反覆间!}


\section{引用测试}

\subsection{参考文献}

测试 \cite{whu-bachelor:1,whu-bachelor:2,whu-bachelor:3,whu-bachelor:5,whu-bachelor:7}

测试 \cite*{whu-bachelor:1,whu-bachelor:2,whu-bachelor:3,whu-bachelor:7}



\section{图表测试}

\begin{figure}[ht]
  \centering
  \includegraphics[width = 5cm]{whu-logo.pdf}
  \caption{武汉大学校徽}
  \label{fig:武汉大学校徽}
\end{figure}

引用图~\ref{fig:武汉大学校徽}

\begin{table}[ht]
  \centering
  \caption{%
    简单的表格和引用 abc 123 %\cite{whu-bachelor:1}
  }
  \label{table:简单的表格}
  \begin{tabular}{cc}
    \hline
    a & b \\ \hline
    c & d \\ \hline
    测试 & 文本 \\ \hline
  \end{tabular}
\end{table}

引用表~\ref{table:简单的表格}


\section{算法}

\begin{algorithm}
\caption{Simulation-optimization heuristic}\label{algorithm}
\KwData{current period $t$, initial inventory $I_{t-1}$, initial capital $B_{t-1}$, demand samples}
\KwResult{Optimal order quantity $Q^{\ast}_{t}$}
$r\leftarrow t$\;
$\Delta B^{\ast}\leftarrow -\infty$\;
\While{$\Delta B\leq \Delta B^{\ast}$ and $r\leq T$}{$Q\leftarrow\arg\max_{Q\geq 0}\Delta B^{Q}_{t,r}(I_{t-1},B_{t-1})$\;
$\Delta B\leftarrow \Delta B^{Q}_{t,r}(I_{t-1},B_{t-1})/(r-t+1)$\;
\If{$\Delta B\geq \Delta B^{\ast}$}{$Q^{\ast}\leftarrow Q$\;
$\Delta B^{\ast}\leftarrow \Delta B$\;}
$r\leftarrow r+1$\;}
\end{algorithm}

引用 \autoref{algorithm}

\section{已定义好的一些数学定理环境}


\begin{definition}[测度]
  (参见文献xxx) 这是一段文字 $E = m c^2$  (中文括号)和 (西文括号)
\end{definition}

\begin{theorem}
  这是一段文字 $E = m c^2$
\end{theorem}


\begin{proof}
  这是一段文字 $E = m c^2$
\end{proof}

\begin{proof}[定理xx的证明]
  这是一段文字 $E = m c^2$
\end{proof}

\begin{example}
  这是一段文字 $E = m c^2$
\end{example}

\begin{property}
  这是一段文字 $E = m c^2$
\end{property}

\begin{proposition}
  这是一段文字 $E = m c^2$
\end{proposition}

\begin{corollary}
  这是一段文字 $E = m c^2$
\end{corollary}

\begin{lemma}
  这是一段文字 $E = m c^2$
\end{lemma}

\begin{axiom}
  这是一段文字 $E = m c^2$
\end{axiom}

\begin{counterexample}
  这是一段文字 $E = m c^2$
\end{counterexample}

\begin{conjecture}
  这是一段文字 $E = m c^2$
\end{conjecture}

\begin{question}
  这是一段文字 $E = m c^2$
\end{question}

\begin{claim}
  这是一段文字 $E = m c^2$
\end{claim}

\begin{remark}
  这是一段文字 $E = m c^2$
\end{remark}

\begin{theorem}[Banach-Steinhaus]\label{thm:test}
  设 $E$ 是 Banach 空间, $F$ 是赋范空间, $(u_i)_{i\in I}$ 是一族从 $E$ 到 $F$ 的有界线性算子,
  即 $(u_i)_{i\in I}\subset \mathcal{B}(E,F)$. 若对每一点 $x\in E$, 有
  $\sup_{i\in I} \|u_i(x)\|<\infty$, 则
  \begin{equation}\label{eq:test1}
    \sup_{i\in I} \|u_i\| < \infty.
  \end{equation}
\end{theorem}

我想引用定理~\ref{thm:test} 和公式~\ref{eq:test1}


定理括号测试:

\begin{theorem}
  测试
  \begin{enumerate}
    \item 中文(括号)没输入空格的效果
    \item 中文 (括号) 输入空格的效果
    \item 西文(括号)没输入空格的效果
    \item 西文 (括号) 输入空格的效果
  \end{enumerate}
\end{theorem} 


\begin{proof}
  test
  \[
    \int_{0}^{1} x^2 \d x
  \]
\end{proof}

\begin{proof}
  test
  \[
    \int_{0}^{1} x^2 \d x \qedhere
  \]
\end{proof}


\section{字体测试}
字体测试:\\
{宋体} {\heiti 黑体} {\kaishu 楷书} {\fangsong 仿宋}\\
{\rmfamily 罗马字族} {\sffamily 无衬线字族} {\ttfamily 打字机}\\
{\bfseries 粗体} {\itshape 意大利} {\slshape 倾斜}

伪粗体测试:\\
{\bfseries\songti 伪粗体} {\bfseries\kaishu 伪粗体} {\bfseries\heiti 伪粗体} {\bfseries\fangsong 伪粗体} {\bfseries 伪粗体}

伪斜体测试:\\
{\itshape\songti 伪斜体} {\itshape\kaishu 伪斜体} {\itshape\heiti 伪斜体} {\itshape\fangsong 伪斜体} {\itshape 伪斜体}

叠加测试:\\
{\bfseries\itshape 伪粗斜体} {\bfseries\sffamily 伪粗黑体} {\bfseries\ttfamily 伪粗仿宋} {\itshape\sffamily 伪斜黑体} {\itshape\ttfamily 伪斜仿宋}

Lorem ipsum dolor sit amet, consectetur adipiscing elit, sed do eiusmod tempor incididunt ut labore et dolore magna aliqua. Ut enim ad minim veniam, quis nostrud exercitation ullamco laboris nisi ut aliquip ex ea commodo consequat. Duis aute irure dolor in reprehenderit in voluptate velit esse cillum dolore eu fugiat nulla pariatur.Lorem ipsum dolor sit amet, consectetur adipiscing elit, sed do eiusmod tempor incididunt ut labore et dolore magna aliqua. Ut enim ad minim veniam, quis nostrud exercitation ullamco laboris nisi ut aliquip ex ea commodo consequat. Duis aute irure dolor in reprehenderit in voluptate velit esse cillum dolore eu fugiat nulla pariatur.
% 当然你也可以直接在这里写,不过这样不太方便管理
\chapter{BBBB}


% 参考文献
% \nocite{*}
\printbibliography

% 发表的与学位论文相关的科研成果目录
% \newenvironment{achievements}{\enumerate[label={[\arabic*]}, leftmargin=*]}{\endenumerate}

\chapter*{攻博期间发表的与学位论文相关的科研成果目录}

\section*{参与的科研项目:}

\begin{achievements}
\item 科研项目1科研项目1科研项目1科研项目1科研项目1
\item 科研项目2科研项目2科研项目2科研项目2科研项目2
\end{achievements}

\section*{发表的学术论文:}

\begin{achievements}
\item 学术论文1学术论文1学术论文1学术论文1学术论文1
\item 学术论文2学术论文2学术论文2学术论文2学术论文2
\end{achievements}

\section*{专利与软件著作权:}

\begin{achievements}
\item 专利与软件著作权1专利与软件著作权1专利与软件著作权1专利与软件著作权1专利与软件著作权1
\item 专利与软件著作权2专利与软件著作权2专利与软件著作权2专利与软件著作权2专利与软件著作权2
\end{achievements}


% 致谢
\begin{acknowledgements}
  致谢是以简短的文字对课题研究与论文撰写过程中直接给予帮助的人员(例如指导教师、答疑教师及其他人员)表示谢意。致谢是对他人劳动的尊重,也是学术规范。内容限一页。
\end{acknowledgements}

% 附录
\appendix

% 附录
\chapter{测试}

\section{公式测试}
\section{公式测试}

\begin{equation}
  a^2 + b^2 = c^2
\end{equation}

\chapter{测试}

\section{公式测试}
\subsection{编号测试}
\subsection{编号测试}
\subsubsection{编号测试}
\subsubsection{编号测试}

\section{公式测试}

\begin{equation}\label{equation:appendix-test}
  a^2 + b^2 = c^2
\end{equation}

公式~\eqref{equation:appendix-test}


\chapter{数据}

\section{第一个测试}
测试公式编号
\begin{equation}
  1+1=2.
\end{equation}

表格编号测试

\begin{table}[h]
  \centering
  \caption{测试表格}
  \begin{tabular}{*{5}c}
    \hline
    11 & 13  & 13  & 13  & 13 \\
    12 & 14  & 13  & 13  & 13 \\
    \hline
  \end{tabular}
\end{table}

\end{document}